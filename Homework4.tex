\documentclass[a4paper,11pt,oneside,onecolumn]{article}
\usepackage[utf8]{inputenc}
\usepackage{cmap}
\usepackage[T1]{fontenc}
\usepackage{beton}
\usepackage{eulervm}
\renewcommand{\sfdefault}{\rmdefault}
\usepackage[pdftex,colorlinks,citecolor=black,filecolor=black,linkcolor=black,urlcolor=black]{hyperref}
\usepackage{tikz}
\usetikzlibrary{positioning,arrows,automata,decorations.pathreplacing,decorations.text,decorations.markings,arrows,shapes,calc,fit}
\usepackage{caption}
\usepackage{subcaption}
\usepackage{multirow}
\usepackage{hhline}
\usepackage{graphicx}
\usepackage{amsmath}
\usepackage[noend]{algpseudocode}

\newcommand{\comment}[1]{%
  \text{\phantom{(#1)}} \tag{#1}
}

\title{Homework 4}
\author{Mitar Milutinovic (24090156)\thanks{Worked together with Shiry Ginosar, Valkyrie Savage, Orianna DeMasi.}}

\renewcommand\thesection{\arabic{section}.}
\renewcommand\thesubsection{\thesection (\alph{subsection})}

\begin{document}

\maketitle

\section{}

\section{}

\section{}

\section{}

\section{}

We designed a voting schema which allows a group of people to better decide on a common opinion about an issue. Currently,
the most used approach is to simply count number of votes against and for, while not taking into account people who do
not cast a vote. Our approach is to have each person define delegates whose votes will be counted when he/she does not
vote him/herself. In this way we get a social network, a trust network, between users which can be
used to transitively compute missing votes. We believe such a result better represents the will of the group.

In the course of the project, we want to compare it with some other similar voting schemas with delegation. We want to visualize
our approach and possibly find a way to visualize differences to other schemas. We will analyze the algorithm for computing the
results according to our voting schema.

On the project I will be working with Valkyrie Savage.

\end{document}
