\documentclass[a4paper,11pt,oneside,onecolumn]{article}
\usepackage[utf8]{inputenc}
\usepackage{cmap}
\usepackage[T1]{fontenc}
\usepackage{beton}
\usepackage{eulervm}
\renewcommand{\sfdefault}{\rmdefault}
\usepackage[pdftex,colorlinks,citecolor=black,filecolor=black,linkcolor=black,urlcolor=black]{hyperref}
\usepackage{tikz}
\usetikzlibrary{positioning,arrows,automata,decorations.pathreplacing,decorations.text,decorations.markings,arrows,shapes,calc,fit}
\usepackage{caption}
\usepackage{subcaption}
\usepackage{multirow}
\usepackage{hhline}
\usepackage{graphicx}
\usepackage{amsmath}
\usepackage[noend]{algpseudocode}

\newcommand{\comment}[1]{%
  \text{\phantom{(#1)}} \tag{#1}
}

\title{Homework 3}
\author{Mitar Milutinovic (24090156)\thanks{Worked together with Shiry Ginosar, Valkyrie Savage, Orianna DeMasi, Christos-Alexandros Psomas.}}

\renewcommand\thesection{\arabic{section}.}
\renewcommand\thesubsection{\thesection (\alph{subsection})}

\begin{document}

\maketitle

\section{}

\subsection{}

Best expert, wrong half of the days: $L^* = \left\lfloor \frac{T}{2} \right\rfloor \le \frac{T}{2}$.
There are two experts, $n=2$, multiplicative weight is $\epsilon=0.5$.

\begin{align*}
L & \le \frac{\log n}{\epsilon} + (1 + \epsilon) L^* \\
 & \le \frac{\log n}{0.5} + (1 + 0.5) \frac{T}{2} \\
 & = 2\log n + \frac{3T}{4} \\
\end{align*}

\subsection{}

Upper bound for expected loss for multiplicative weight framework is $L \le \frac{\log n}{\epsilon} + (1 + \epsilon) L^*$.
Using best expert loss $L^* = \left\lfloor \frac{T}{2} \right\rfloor \le \frac{T}{2}$ we get:

\begin{align*}
L & \le \frac{\log n}{\epsilon} + (1 + \epsilon) L^* \\
 & \le \frac{\log n}{\epsilon} + (1 + \epsilon) \frac{T}{2} \\
\end{align*}

\section{}

I used a modified version of the provided code and I understand it, it is straightforward. Using the game matrix:

$$
A = \left[\begin{array}{ccc}
0 & 1 & -1 \\
-1 & 0 & 1 \\
1 & -1 & 0 \\
0 & -1 & -1 \\
\end{array}\right]
$$

Translated and scaled matrix:

$$
A' = \left[\begin{array}{ccc}
0.5 & 1 & 0 \\
0 & 0.5 & 1 \\
1 & 0 & 0.5 \\
0.5 & 0 & 0 \\
\end{array}\right]
$$

For $T=10$, row's player strategy is:

$$
\left[\begin{array}{cccc}
0.2527 & 0.3288 & 0.1658 & 0.2527 \\
\end{array}\right]
$$

Column's player strategy is:

$$
\left[\begin{array}{ccc}
0.7273 & 0.0909 & 0.1818 \\
\end{array}\right]
$$

$\delta$-equilibrium:

$$
\delta = 0.2131
$$

For $T=100$, row's player strategy is:

$$
\left[\begin{array}{cccc}
0.1731 & 0.3258 & 0.0046 & 0.4965 \\
\end{array}\right]
$$

Column's player strategy is:

$$
\left[\begin{array}{ccc}
0.6040 & 0.0990 & 0.2970 \\
\end{array}\right]
$$

$\delta$-equilibrium:

$$
\delta = 0.0496
$$

For $T=139$, row's player strategy is:

$$
\left[\begin{array}{cccc}
0.1592 & 0.3328 & 0.0007 & 0.5073 \\
\end{array}\right]
$$

Column's player strategy is:

$$
\left[\begin{array}{ccc}
0.6214 & 0.0786 & 0.3000 \\
\end{array}\right]
$$

$\delta$-equilibrium:

$$
\delta = 0.0412
$$

\section{}

We are using the same nomenclature as used when we showed a $1 + \epsilon$ approximate solution to the fractional path
routing problem in $O(k^2m\frac{\log^2 m}{\epsilon^2})$ time using the experts framework. For
$O(km\frac{\log^2 m}{\epsilon^2})$ time we change our algorithm so that each day we uniformly randomly chose only one
pair $(s_i, t_i)$ and find new shortest path (for new tolls for that day) for that pair. Number of days we have to run the
algorithm prolongs to $T \le \frac{8k \log k}{\epsilon ^2}$ to achieve the desired $1+\delta$ approximate solution, but we have to
find only one shortest path each day instead of $k$, thus running time improves for a factor of $k$.
Because we are routing only one path per day with just one unit of toll, total gain per day is $G_t \le 1$. Thus,
both mean and variance is at most $1$ (and $k$). Based on lemma 1 each pair is routed $\frac{T}{k}(1 \pm \epsilon)$ times
with probability at least $1 - \frac{1}{k}$. Over all days $G=C^*$, so $E[G_t] \approx \frac{C^*}{k}$. We can sum over all days to
obtain $E[G] \le \frac{TC^*}{k} $.

\begin{align*}
G & \ge (1-\epsilon) G^* - \frac{\log n}{\epsilon} \\
\comment{by lemma 2} \epsilon E(G) & \ge G - E[G] \\
\epsilon E(G) \ge G - E[G] & \ge (1-\epsilon) G^* - \frac{\log n}{\epsilon} - E[G] \\
\epsilon E(G) \ge (1 + \epsilon) E[G] & \ge (1-\epsilon) G^* - \frac{\log n}{\epsilon} \\
\comment{$E[G] \le \frac{TC^*}{k} $} (1 + \epsilon) \frac{T C^*}{k} & \ge (1-\epsilon) G^* - \frac{\log n}{\epsilon} \\
(1 + \epsilon) \frac{T C^*}{k} & \ge (1-\epsilon) \frac{TC_{max}}{k} - \frac{\log n}{\epsilon} \\
(1 + \epsilon) T C^* & \ge (1-\epsilon) TC_{max} - \frac{k \log n}{\epsilon} \\
\frac{k \log n}{\epsilon} & \ge (1-\epsilon) TC_{max} - (1 + \epsilon) T C^* \\
\frac{k \log n}{\epsilon} & \ge T ( (1-\epsilon) C_{max} - (1 + \epsilon) C^* ) \\
\comment{by lemma 1} \frac{k \log n}{\epsilon} & \ge \frac{8k \log n}{e^2} \left( (1-\epsilon) C_{max} - (1 + \epsilon) C^* \right) \\
1 & \ge \frac{8}{e} \left( (1-\epsilon) C_{max} - (1 + \epsilon) C^* \right) \\
\epsilon + 8 (1 + \epsilon) C^* & \ge 8 (1-\epsilon) C_{max} \\
\frac{\epsilon + 8 (1 + \epsilon) C^*}{8 (1-\epsilon) } & \ge C_{max} \\
\left( 1 + \frac{17 \epsilon }{8 (1-\epsilon) } \right) C^* & \ge C_{max} \\
\end{align*}

Thus:

$$
\delta = 1 + \frac{17 \epsilon }{8 (1-\epsilon) }
$$

\section{}

\subsection{}

Because adversary (for which we assume is capable of influencing the market and any other inputs the algorithm is getting)
knows the deterministic algorithm and its logic, adversary can create conditions which make the player's
deterministic algorithm decide to invest in the instrument. But in the moment that it invests, adversary can make the market
yield zero dollars for that instrument. For example, where there is only one time interval ($T=1$) and two instruments ($n=2$), adversary can
simply make deterministically chosen instrument to yield zero dollars and the other one, not chosen, $P$.

\end{document}
