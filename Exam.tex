\documentclass[a4paper,11pt,oneside,onecolumn]{article}
\usepackage[utf8]{inputenc}
\usepackage{cmap}
\usepackage[T1]{fontenc}
\usepackage{beton}
\usepackage{eulervm}
\renewcommand{\sfdefault}{\rmdefault}
\usepackage[pdftex,colorlinks,citecolor=black,filecolor=black,linkcolor=black,urlcolor=black]{hyperref}
\usepackage{tikz}
\usetikzlibrary{positioning,arrows,automata,decorations.pathreplacing,decorations.text,decorations.markings,arrows,shapes,calc,fit}
\usepackage{caption}
\usepackage{subcaption}
\usepackage{multirow}
\usepackage{hhline}
\usepackage{graphicx}
\usepackage{amsmath}
\usepackage[noend]{algpseudocode}

\newcommand{\comment}[1]{%
  \text{\phantom{(#1)}} \tag{#1}
}

\title{Exam}
\author{Mitar Milutinovic (24090156)}

\renewcommand\thesection{\arabic{section}.}
\renewcommand\thesubsection{\thesection (\alph{subsection})}
\renewcommand\thesubsubsection{\thesubsection (\roman{subsubsection})}

\def\zline{
    \vspace{-2.9em}
    \begin{center}\centering\line(1,0){150}\end{center}
    \vspace{-2.9em}
}

\begin{document}

\maketitle

\section{}

\subsection{}

Optimal fractional solution has $\frac{4}{3}$ load on every edge. We route 1 unit of load between:
\begin{itemize}
\item $v_1$ and $v_2$: $0.5$ along $v_1-v_4-v_2$, $0.5$ along $v_1-v_5-v_2$
\item $v_2$ and $v_3$: $0.5$ along $v_2-v_5-v_3$, $0.5$ along $v_2-v_4-v_3$
\item $v_1$ and $v_3$: $0.5$ along $v_1-v_4-v_3$, $0.5$ along $v_1-v_5-v_3$
\item $v_4$ and $v_5$: $\frac{1}{3}$ along $v_4-v_2-v_5$, $\frac{1}{3}$ along $v_4-v_1-v_5$, $\frac{1}{3}$ along $v_4-v_3-v_5$
\end{itemize}

Because all loads on all edges are equal and equal to maximum load it is not possible that there would be a better solution.
Lowering load on one edge would have to increase load on some other edge which would increase overall maximum load.

\subsection{}

Because graph contains even number of nodes we do not have to add any dummy node to be able to satisfy Edmonds' condition and
use odd-set constraints:

\begin{align*}
\max \sum_e w_e x_e &  \\
\forall v \in V: \sum_{e=(u,v)} x_e & \le 1 \\
\forall S \subseteq V, |S| \textrm{ is odd}: \sum_{e=(u,v) \in (S \times (V \setminus S))} x_e & \ge 1 \\
x_e & \ge 0 \\
\end{align*}

For the dual, we introduce new variables for nontrivial constraints from the primal. For the first constraint, for each
vertex $v$ constraint, we introduce variable $y_v$. For the second constraint, for each $S$ constraint, we introduce variable
$y_S$. The resulting linear program is then:

\begin{align*}
\min \sum_v y_v + \sum_{S \subseteq V, |S| \textrm{ is odd}} y_S & \\
\forall e = (u,v) \in E: y_u + y_v - \sum_{S \subseteq V, |S| \textrm{ is odd}, u \in S, v \in (V \setminus S)} y_S & \ge w_e \\
y_v & \ge 0 \\
y_S & \ge 0 \\
\end{align*}

Solution for primal, maximum weight matching, is $5$, edges: $(v_1, v_3)$, $(v_2, v_5)$, $(v_4, v_6)$. Solution for dual, vertex
cover, is $6$, $\forall v \in V: p(v) = 1$.

\subsection{}

We assume that such separating hyperplane with margin $\gamma$ exists. We run perceptron algorithm on given points and remember mistakes. If there are no
mistakes, we output hyperplane. Otherwise we duplicate points which were mistakes and rerun our algorithm on this extended set of
points with initial hyperplane set to currently found hyperplane.

\subsection{}

The dual of the linear program $Ax \le b$, $\min cx$ is $y'A \le c$, $\max y'b$.

\begin{align*}
\min cx & \\
Ax & \le b \\
x & \ge 0 \\
\end{align*}

\zline

\begin{align*}
\max -cx & \\
Ax & \le b \\
x & \ge 0 \\
\end{align*}

\zline

\begin{align*}
\min y^Tb & \\
y^TA & \ge -c \\
y & \ge 0 \\
\end{align*}

\zline

\begin{align*}
\min y^Tb & \\
-y^TA & \le c \\
y & \ge 0 \\
\end{align*}

\zline

\begin{align*}
\max y'^Tb & \\
y'^TA & \le c \\
y' & \le 0 \\
\end{align*}

\subsection{}

\subsubsection{}

Not true. If $y = 1$, $A = 1$, $c = 1$, $Ax$ can be $1$ with $x = 1$ and $b = 1$.

\subsubsection{}

Not true. If $x = 0$, $b = 0$, $A = 0$, $c$ can be $1$ while $y^TA$ is $0$.

\subsubsection{}

True. Negation of complementary slackness.

\subsubsection{}

Not true. I believe condition for primal is missing.

\subsection{}

Hyperplane is:
$$
\sum_p p \cdot l(p)
$$

So the normal is average/sum of all points for positive label and opposite (negated) points for negative labels.

\section{}

\section{}

\section{}

\section{}

\end{document}
