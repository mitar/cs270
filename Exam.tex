\documentclass[a4paper,11pt,oneside,onecolumn]{article}
\usepackage[utf8]{inputenc}
\usepackage{cmap}
\usepackage[T1]{fontenc}
\usepackage{beton}
\usepackage{eulervm}
\renewcommand{\sfdefault}{\rmdefault}
\usepackage[pdftex,colorlinks,citecolor=black,filecolor=black,linkcolor=black,urlcolor=black]{hyperref}
\usepackage{tikz}
\usetikzlibrary{positioning,arrows,automata,decorations.pathreplacing,decorations.text,decorations.markings,arrows,shapes,calc,fit}
\usepackage{caption}
\usepackage{subcaption}
\usepackage{multirow}
\usepackage{hhline}
\usepackage{graphicx}
\usepackage{amsmath}
\usepackage[noend]{algpseudocode}

\newcommand{\comment}[1]{%
  \text{\phantom{(#1)}} \tag{#1}
}

\title{Exam}
\author{Mitar Milutinovic (24090156)}
\date{}

\renewcommand\thesection{\arabic{section}.}
\renewcommand\thesubsection{\thesection (\alph{subsection})}
\renewcommand\thesubsubsection{\thesubsection (\roman{subsubsection})}

\def\zline{
    \vspace{-2.9em}
    \begin{center}\centering\line(1,0){150}\end{center}
    \vspace{-2.9em}
}

\begin{document}

\maketitle

\section{}

\subsection{}

Optimal fractional solution has $\frac{4}{3}$ load on every edge. We route 1 unit of load between:
\begin{itemize}
\item $v_1$ and $v_2$: $0.5$ along $v_1-v_4-v_2$, $0.5$ along $v_1-v_5-v_2$
\item $v_2$ and $v_3$: $0.5$ along $v_2-v_5-v_3$, $0.5$ along $v_2-v_4-v_3$
\item $v_1$ and $v_3$: $0.5$ along $v_1-v_4-v_3$, $0.5$ along $v_1-v_5-v_3$
\item $v_4$ and $v_5$: $\frac{1}{3}$ along $v_4-v_2-v_5$, $\frac{1}{3}$ along $v_4-v_1-v_5$, $\frac{1}{3}$ along $v_4-v_3-v_5$
\end{itemize}

Because all loads on all edges are equal and equal to maximum load it is not possible that there would be a better solution.
Lowering load on one edge would have to increase load on some other edge which would increase overall maximum load.

\subsection{}

Because graph contains even number of nodes we do not have to add any dummy node to be able to satisfy Edmonds' condition and
use odd-set constraints:

\begin{align*}
\max \sum_e w_e x_e &  \\
\forall v \in V: \sum_{e=(u,v)} x_e & \le 1 \\
\forall S \subseteq V, |S| \textrm{ is odd}: \sum_{e=(u,v) \in (S \times (V \setminus S))} x_e & \ge 1 \\
x_e & \ge 0 \\
\end{align*}

For the dual, we introduce new variables for nontrivial constraints from the primal. For the first constraint, for each
vertex $v$ constraint, we introduce variable $y_v$. For the second constraint, for each $S$ constraint, we introduce variable
$y_S$. The resulting linear program is then:

\begin{align*}
\min \sum_v y_v + \sum_{S \subseteq V, |S| \textrm{ is odd}} y_S & \\
\forall e = (u,v) \in E: y_u + y_v - \sum_{S \subseteq V, |S| \textrm{ is odd}, u \in S, v \in (V \setminus S)} y_S & \ge w_e \\
y_v & \ge 0 \\
y_S & \ge 0 \\
\end{align*}

Solution for primal, maximum weight matching, is $5$, edges: $(v_1, v_3)$, $(v_2, v_5)$, $(v_4, v_6)$. Solution for dual, vertex
cover, is $6$, $\forall v \in V: p(v) = 1$.

\subsection{}

We assume that such separating hyperplane with margin $\gamma$ exists. We run perceptron algorithm on given points and remember mistakes. If there are no
mistakes, we output hyperplane. Otherwise we duplicate points which were mistakes and rerun our algorithm on this extended set of
points with initial hyperplane set to currently found hyperplane.

\subsection{}

The dual of the linear program $Ax \le b$, $\min cx$ is $y'A \le c$, $\max y'b$.

\begin{align*}
\min cx & \\
Ax & \le b \\
x & \ge 0 \\
\end{align*}

\zline

\begin{align*}
\max -cx & \\
Ax & \le b \\
x & \ge 0 \\
\end{align*}

\zline

\begin{align*}
\min y^Tb & \\
y^TA & \ge -c \\
y & \ge 0 \\
\end{align*}

\zline

\begin{align*}
\min y^Tb & \\
-y^TA & \le c \\
y & \ge 0 \\
\end{align*}

\zline

\begin{align*}
\max y'^Tb & \\
y'^TA & \le c \\
y' & \le 0 \\
\end{align*}

\subsection{}

\subsubsection{}

Not true. If $y = 1$, $A = 1$, $c = 1$, $Ax$ can be $1$ with $x = 1$ and $b = 1$.

\subsubsection{}

Not true. If $x = 0$, $b = 0$, $A = 0$, $c$ can be $1$ while $y^TA$ is $0$.

\subsubsection{}

True. Negation of complementary slackness.

\subsubsection{}

Not true. We believe condition for primal is missing.

\subsection{}

Hyperplane is:
$$
\sum_p p \cdot l(p)
$$

So the normal is average/sum of all points for positive label and opposite (negated) points for negative labels.

\section{}

\section{}

\subsection{}

$$
T = O(X)
$$

$$
\mu = \tilde\Omega(X)
$$

\subsection{}

$$
T = \Omega(X)
$$

$$
\mu = \tilde O(X)
$$

\subsection{}

$$
\mu = \tilde\Omega\left(\frac{h(G)^2}{2}\right) = \tilde\Omega\left(h(G)^2\right)
$$

\subsection{}

$$
\mu = \tilde O\left(2 h(G) \right) = \tilde O\left( h(G) \right)
$$

\subsection{}

\subsubsection{}

$$
h(S) = \frac{|E(S,V-S)|}{n \min\left(|S|,|V-S|\right)} = \frac{k(n-k)}{nk} = \frac{n-k}{n}
$$
where $k = |S|$ and $|S| \le \frac{|V|}{2}$.

$$
h(G) = \min_{S\subseteq V} h(S) = \frac{\frac{n}{2}}{n} = \frac{1}{2}
$$

$$
\mu = O\left(2 h(G) \right) = O \left(1\right)
$$

This also agrees with intuition that because a clique is highly connected, $\mu$ should be close to 1.

\section{}

\subsection{}

$0$, because rows cancel each other out, on average, because we are multiplying each row by $+1$ and $-1$ with equal probability.

\subsection{}

$\frac{d}{k}$, because signal $d$ is lost among $k$ other rows.

\subsection{}

Probably it is assumed non-diagonal entries, because diagonal entries have a value of $d$, squared $d^2$.

\subsection{}

\subsection{}

We can compute $\tilde A$ and $\tilde A \tilde A^T$ as described and sort in increasing order all non-diagonal entries by their absolute value into list $L$. For each entry we know corresponding rows from $A$ involved in its computation. For each row from $A$ we add to its counter variable corresponding entry's position in $L$. We repeat this for multiple random partitions of $A$. We output two rows with
highest count in their counter variables.

\section{}

Multicommodity flow on instance on $G$ with demand graph $K_{|V|}$ routes one unit of flow between each pair of vertices in $G$.

$\alpha$, the maximum congestion of any edge has a corresponding edge $e_\alpha$. Congestion on this edge is $0 < \alpha \le n$, where $n = |V|$. $\phi(G)$ has
a corresponding cut with with minimum sparsity $\phi(G) = \frac{|E(C,\bar C)|}{|C||\bar C|}$. Without the loss in generality we can assume $|C| \le |\bar C|$. We want to show that $f(\alpha) \le \phi(G)$. This is trivial because $\phi(G)$ is non-negative and by setting $f(\alpha)$ to a constant function $0$ we have a lower bound.

To improve this lower bound we observe that $e_\alpha \in E(C,\bar C)$. Multicommodity flow routes $|C||\bar C|$ units of flow between all nodes in $C$ and all nodes in $\bar C$ over edges in $E(C,\bar C)$. If $|E(C,\bar C)| = 1$ then $E(C,\bar C) = \{e_\alpha\} $ and $\alpha = |C||\bar C|$ and $\frac{1}{\alpha} = \frac{1}{|C||\bar C|} \le \frac{|E(C,\bar C)|}{|C||\bar C|} = \frac{1}{|C||\bar C|}$. Otherwise, all flow between $C$ and $\bar C$ goes over multiple edges in $E(C,\bar C)$ and for that there are two extremes: all flow goes still over one edge and all flow is equally distributed over all edges:

$$
\frac{|C||\bar C|}{|E(C,\bar C)|} \le \alpha \le |C||\bar C|
$$

We plug this into $\frac{1}{\alpha}$ and get:

$$
\frac{1}{|C||\bar C|} \le \frac{|E(C,\bar C)|}{|C||\bar C|}
$$
and:
$$
\frac{|E(C,\bar C)|}{|C||\bar C|} \le \frac{|E(C,\bar C)|}{|C||\bar C|}
$$

By that we have shown that $\frac{1}{\alpha}$ is a better lower bound
for sparsity of the graph.

\end{document}
